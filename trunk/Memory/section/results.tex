\chapter{Resultados}
El desarrollo de este juego ha permitido aprender una gran cantidad de t�cnicas y conocimientos en el desarrollo de videojuegos, tanto de arte como de programaci� y dise�o.

El resultado final del juego se debe al trabajo hecho durante un a�o. Se ha conseguido una buena sensaci�n de estar en un bosque nevado dado la complejidad del escenario. Recrear un entorno natural es complicado. 

Como se puede apreciar en la \ref{fig:game_2}, las sombras del juego son de gran calidad, aportando bastante calidad gr�fica. Tambi�n se puede apreciar el trabajo hecho con Caperucita y un poco de las animaciones vienda la posici�n en que se est� moviendo.

En la figura \ref{fig:game_1} se puede ver el dise�o de los enemigos y del personaje, as� como tambi�n alguna de las part�culas dise�adas para el juego. Con los enemigos y Caperucita se puede ver la posici�n en las animaciones.

En la figura \ref{fig:game_3} se puede apreciar el efecto de \emph{Motion Blur} al hacer el ataque contra un conejo.

Se puede ver en la \ref{fig:game_4} el trabajo hecho en el puente y tambi�n en las part�culas que simulan una cascada de agua.

El trabajo hecho en el HUD, tanto de Caperucita como del Lobo, y en las part�culas de sangre se puede observar en la figura \ref{fig:game_5}. Se han creado unas part�culas que generen mucha sangre para buscar esa sensaci�n \emph{gore} que se plante� en el inicio del desarrollo del juego y que ha perdurado en todo su desarrollo.

Finalmente, en la figura \ref{fig:game_6} se puede ver el Lobo despu�s de llamar a otros enemigos para que hagan acto de presencia. Se ven las part�culas de nieve que genera el Lobo despu�s de hacer esta acci�n.

\begin{figure}
\begin{center}
\includegraphics[scale=0.25]{figures/results/game_2.eps}
\end{center}
\emph{\caption{Captura del juego}\label{fig:game_2}}
\end{figure}

\begin{figure}
\begin{center}
\includegraphics[scale=0.25]{figures/results/game_1.eps}
\end{center}
\emph{\caption{Combate contra dos tipos de enemigo}\label{fig:game_1}}
\end{figure}

\begin{figure}
\begin{center}
\includegraphics[scale=0.25]{figures/results/game_3.eps}
\end{center}
\emph{\caption{Efecto de Motion Blur mientras se mata a un enemigo}\label{fig:game_3}}
\end{figure}

\begin{figure}
\begin{center}
\includegraphics[scale=0.25]{figures/results/game_4.eps}
\end{center}
\emph{\caption{Puente y cascada}\label{fig:game_4}}
\end{figure}

\begin{figure}
\begin{center}
\includegraphics[scale=0.25]{figures/results/game_5.eps}
\end{center}
\emph{\caption{Combate contra enemigos}\label{fig:game_5}}
\end{figure}


\begin{figure}
\begin{center}
\includegraphics[scale=0.25]{figures/results/game_6.eps}
\end{center}
\emph{\caption{Lobo despu�s de llamar a m�s enemigos}\label{fig:game_6}}
\end{figure}
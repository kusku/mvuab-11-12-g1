\chapter{Conclusiones}

\begin{itemize}

\item Se ha dise�ado e implementado un juego \emph{hack 'n' slash} sobre el cuento de Caperucita Roja, creando una reproducci�n del cuento distinta en una ambientaci�n oscura y \emph{gore}.

\item Se ha creado una inteligencia artificial pensada y orientada al combate cuerpo a cuerpo, teniendo en cuenta que cada personaje con inteligencia no est� solo en el escenario sino con m�s entidades com �l.

\item Se ha dise�ado un escenario grande para el combate, con sensaci�n de estar en un bosque nevado abierto, pero al mismo tiempo estar limitado a deambular por zonas concretas, limitando as� todo el acceso al bosque.

\item Se ha implementado un motor de juego gen�rico con tecnolog�as que se utilizan actualmente en el desarrollo de videojuegos, como por ejemplo Direct, Wwise o PhysX.

\item Se han encontrado problemas en el momento de desarrollar con Lua / Luabind. Al crear una clase de Lua en C++ que esta heredaba de c�digo de C++, en alg�n momento aleatorio de la ejecuci�n, Lua perd�a esas referencias haciendo que el juego dejara de funcionar. Se solucion� escribiendo todo el c�digo de nuevo en C++.

\item Se han encontrado problemas en la reserva de memoria de Lua. Siempre se iba reservado m�s y m�s memoria hasta agotar toda la memoria del ordenador en que se ejecutaba el juego. Se solucion� pasando el c�digo de Lua a C++.

\item Se han encontrado bastantes problemas con el exportador de Cal3D y 3ds Max 2012. El exportador muchas veces o, dejaba el programa de modelado inutilizado, o generaba ficheros corruptos, modificando as� toda la malla que se exportaba.

\item El escenario se ha tenido que rehacer m�s de una vez por motivos de dise�o, ya que las versiones anteriores no cumplian los objetivos de jugabilidad.

\item Al no tener un editor de niveles completo, la ubicaci�n de los �rboles se hizo manualmente, lo cual result� ser una tarea compleja y lenta, gastando bastante tiempo.

\item Animar un personaje con ropa es mucho m�s complejo de lo esperado, y tom� m�s tiempo e inclusive modificar el dise�o original del personaje principal.

\end{itemize}